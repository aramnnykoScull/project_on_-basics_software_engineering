\documentclass[14pt,a4paper]{extarticle}
\usepackage[T2A]{fontenc}
\usepackage[utf8]{inputenc}
\usepackage[russian]{babel}
\usepackage{geometry}
\usepackage{titlesec}
\usepackage{setspace}
\usepackage{graphicx}
\usepackage{indentfirst}
\usepackage{enumitem}
\usepackage{float}
\usepackage{listings}
\usepackage{caption} % Для настройки подписей

% Настройки по ГОСТ
\geometry{left=30mm, right=15mm, top=20mm, bottom=20mm}
\onehalfspacing
\parindent=1.25cm

% Используем шрифт Times New Roman
\usepackage{mathptmx}

% Настройка стилей заголовков по ГОСТ
\titleformat{\section}{\normalfont\bfseries\centering}{\thesection}{1em}{}
\titleformat{\subsection}{\normalfont\bfseries\raggedright}{\thesubsection}{1em}{}
\titleformat{\subsubsection}{\normalfont\bfseries\raggedright}{\thesubsubsection}{1em}{}

% Настройка для листингов - подписи снизу
\lstset{
    language=C++,
    basicstyle=\ttfamily\footnotesize,
    numbers=left,
    numberstyle=\tiny,
    frame=single,
    breaklines=true,
    breakatwhitespace=true,
    showspaces=false,
    showstringspaces=false,
    captionpos=b, % Подпись снизу
    inputencoding=utf8,
    extendedchars=true,
    keepspaces=true,
    literate={а}{{\selectfont\char224}}1
             {б}{{\selectfont\char225}}1
             {в}{{\selectfont\char226}}1
             {г}{{\selectfont\char227}}1
             {д}{{\selectfont\char228}}1
             {е}{{\selectfont\char229}}1
             {ё}{{\"e}}1
             {ж}{{\selectfont\char230}}1
             {з}{{\selectfont\char231}}1
             {и}{{\selectfont\char232}}1
             {й}{{\selectfont\char233}}1
             {к}{{\selectfont\char234}}1
             {л}{{\selectfont\char235}}1
             {м}{{\selectfont\char236}}1
             {н}{{\selectfont\char237}}1
             {о}{{\selectfont\char238}}1
             {п}{{\selectfont\char239}}1
             {р}{{\selectfont\char240}}1
             {с}{{\selectfont\char241}}1
             {т}{{\selectfont\char242}}1
             {у}{{\selectfont\char243}}1
             {ф}{{\selectfont\char244}}1
             {х}{{\selectfont\char245}}1
             {ц}{{\selectfont\char246}}1
             {ч}{{\selectfont\char247}}1
             {ш}{{\selectfont\char248}}1
             {щ}{{\selectfont\char249}}1
             {ъ}{{\selectfont\char250}}1
             {ы}{{\selectfont\char251}}1
             {ь}{{\selectfont\char252}}1
             {э}{{\selectfont\char253}}1
             {ю}{{\selectfont\char254}}1
             {я}{{\selectfont\char255}}1
             {А}{{\selectfont\char192}}1
             {Б}{{\selectfont\char193}}1
             {В}{{\selectfont\char194}}1
             {Г}{{\selectfont\char195}}1
             {Д}{{\selectfont\char196}}1
             {Е}{{\selectfont\char197}}1
             {Ё}{{\"E}}1
             {Ж}{{\selectfont\char198}}1
             {З}{{\selectfont\char199}}1
             {И}{{\selectfont\char200}}1
             {Й}{{\selectfont\char201}}1
             {К}{{\selectfont\char202}}1
             {Л}{{\selectfont\char203}}1
             {М}{{\selectfont\char204}}1
             {Н}{{\selectfont\char205}}1
             {О}{{\selectfont\char206}}1
             {П}{{\selectfont\char207}}1
             {Р}{{\selectfont\char208}}1
             {С}{{\selectfont\char209}}1
             {Т}{{\selectfont\char210}}1
             {У}{{\selectfont\char211}}1
             {Ф}{{\selectfont\char212}}1
             {Х}{{\selectfont\char213}}1
             {Ц}{{\selectfont\char214}}1
             {Ч}{{\selectfont\char215}}1
             {Ш}{{\selectfont\char216}}1
             {Щ}{{\selectfont\char217}}1
             {Ъ}{{\selectfont\char218}}1
             {Ы}{{\selectfont\char219}}1
             {Ь}{{\selectfont\char220}}1
             {Э}{{\selectfont\char221}}1
             {Ю}{{\selectfont\char222}}1
             {Я}{{\selectfont\char223}}1
             {«}{{\guillemotleft}}1
             {»}{{\guillemotright}}1
             {№}{{\No}}1,
}

% Настройка нумерации листингов
\renewcommand{\lstlistingname}{Листинг}
\renewcommand{\lstlistlistingname}{Список листингов}







% ------------------------------------------------------------------------------------
% ------------------------------------------------------------------------------------
% ------------------------------------------------------------------------------------
% ------------------------------------------------------------------------------------



\begin{document}

\subsection*{Цель работы:}

\subsection*{Задание:}
\begin{enumerate}[label=\arabic*)]
    \item 
    \item 
    \item 
\end{enumerate}

\section*{Ход работы}

\subsection*{Теоретическая информация}

\subsection*{Подзадание №1}

Код программы представлен ниже (Листинги \ref{lst:point1}, \ref{lst:point_all_h} и \ref{lst:point_all_cpp}, рис. \ref{fig:square}).

\begin{lstlisting}[caption={Код программы для вычисления площади квадрата}, label=lst:point1]
#include "point_all.h"

int main() {
    int side;
    if (!read_int("Введите сторону квадрата: ", side))
        return 1;

    cout << "Площадь квадрата: " << side * side << endl;
    return 0;
}
\end{lstlisting}

\begin{figure}[H]
\centering
\includegraphics[width=0.5\textwidth, height=0.5\textheight, keepaspectratio]{media/image1.png}
\caption{Пример выполнения программы для вычисления площади квадрата}
\label{fig:square}
\end{figure}

\subsection*{Подзадание №2}

Код программы представлен ниже (Листинги \ref{lst:point2}, \ref{lst:point_all_h} и \ref{lst:point_all_cpp}, рис. \ref{fig:volume}).

\begin{lstlisting}[caption={Код программы для вычисления объема прямоугольного параллелепипеда}, label=lst:point2]
#include "point_all.h"

int main() {
    int side_a, side_b, side_h;
    cout << "Для подсчета объема введите три стороны:" << endl;
    
    if (!read_int("Введите сторону a (длина): ", side_a) ||
        !read_int("Введите сторону b (ширина): ", side_b) ||
        !read_int("Введите сторону h (высота): ", side_h))
        return 1;
    
    cout << "Объем: " << side_a * side_b * side_h << endl;
    return 0;
}
\end{lstlisting}

\begin{figure}[H]
\centering
\includegraphics[width=0.7\textwidth, height=0.7\textheight, keepaspectratio]{media/image2.png}
\caption{Пример выполнения программы для вычисления объема}
\label{fig:volume}
\end{figure}

\subsection*{Подзадание №3}

Код программы представлен ниже (Листинги \ref{lst:point3}, \ref{lst:point_all_h} и \ref{lst:point_all_cpp}, рис. \ref{fig:distance}).

\begin{lstlisting}[caption={Код программы для вычисления пройденного расстояния}, label=lst:point3]

\end{lstlisting}

\begin{figure}[H]
\centering
\includegraphics[width=0.8\textwidth, height=0.8\textheight, keepaspectratio]{media/image3.png}
\caption{Пример выполнения программы для вычисления расстояния}
\label{fig:distance}
\end{figure}

\subsection*{Вспомогательные файлы}
Для всех программ использован общий заголовочный файл и функция для чтения и проверки положительного числа.

\begin{lstlisting}[caption={Заголовочный файл point\_all.h}, label=lst:point_all_h]

\end{lstlisting}

\begin{lstlisting}[caption={Файл реализации point\_all.cpp}, label=lst:point_all_cpp]

\end{lstlisting}

\section*{Вывод}

\end{document}